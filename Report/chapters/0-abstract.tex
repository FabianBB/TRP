\begin{abstract}\label{abstract}
This paper studies the use of autonomous guided vehicles (AGVs) in vertical insect farms as a sustainable solution for addressing global issues such as climate change and food security. The study aims to configure an AGV capable of transporting payloads in a simulated miniaturized insect farm in a synchronized network, with a focus on flexibility and ease of use for non-experts. The AGV design proposed utilizes a 360° LIDAR sensor and an RGB camera, and experiments were conducted to evaluate its effectiveness in enabling automated insect farming. The goal is to produce an AGV configuration that is controllable by non-experts and allows the customer to understand how AGVs (with other robots and technologies) can enable automated insect farms, thus stimulating the use of AGVs in insect farms. This study is part of a larger research project with multiple teams and is being developed in cooperation with a customer company. The results of the experiments conducted show that the AGV design is effective in transporting payloads, with the AGV being able to navigate within a pre-mapped farm efficiently and accurately. The use of the 360° LIDAR sensor was found to be crucial in allowing the AGV roam the farm and detect obstacles. Additionally, the AGV was found to be flexible and adjustable to different surfaces, with the ability to adapt to changes in the farm layout.
\end{abstract}

\begin{IEEEkeywords}
AGV, Insect farm, Navigation, Localization, ROS
\end{IEEEkeywords}