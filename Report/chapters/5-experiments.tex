\section{Experiments}
% This section provides a complete and exhaustive description of all 
% experiments/analyses/simulations  you  have  performed  in  order  to  test  or  compare  algorithm 
% performance. Experiments should be reported in a logical sequence. It is important to motivate 
% why  those  experiments have  been  chosen,  and  how  they  will  help  to  answer  the  research 
% question. Experimenting is not about testing the correctness of your software. You have to design 
% experiments that answer your research questions to convince the reader that your answer to the 
% research questions  is correct  (e.g.,  you  may  want to  experiment  on  how much  more  time  your 
% algorithm takes when increasing the complexity of your input).   
% The level of details and information provided in this section should be such that a reader could 
% reproduce all experiments without any inputs from the authors of the report, and without having 
% to  refer  to  any  external  source.  Thus,  you  need  to  be  clear  about  what  settings  (simulation 
% runs, inputs, ...) you have used for each of the experiments.  

To find answers to the research questions proposed earlier, experiments were performed with the Elephant Robotics MyAGV, which is equipped with a built-in Raspberry Pi 4B. The main goal of these experiments was to see how the AGV navigates, maps environments and detects objects.
The following experiments were performed with this AGV:
\begin{itemize}
    \item Mapping experiment
    \item Navigation experiment
    \item LIDAR experiment
    \item Driving surface test
\end{itemize}

\subsection{Mapping experiment}
This experiment revolves around testing the mapping accuracy of the AGV. An enclosed space was created for the AGV to map out entirely. This map is then analyzed to determine the accuracy compared to the real environment.
The map consists of 5 areas named A to E as displayed in Figure \ref{fig:app:grid}. These areas are measured in centimeters of the real map and compared to the pixel amount in the digital map.

\begin{figure}[H]
    \centering
    \includegraphics[width=0.4\textwidth,height=0.45\textheight,keepaspectratio]{images/Map_grid.png}
    \caption{The grid.}
    \label{fig:app:grid}
\end{figure}

\subsection{Navigation experiment}
A navigation experiment was performed using the ROS-navigation packages. This allows for a 2D estimate of the robot's location and a 2D navigation goal toward a given point. The entire map is mapped out in advance in order for the AGV to accurately estimate its position and to perform path planning. This experiment was done using RVIZ and by inputting coordinates into a Python script designed for navigation.

\subsection{LIDAR experiment}
This experiment was performed in order to assess the AGV's object detection capabilities. As mentioned earlier, one manner by which the AGV is able to detect objects is by using its built-in LIDAR sensor which is placed roughly 10 cm above the ground. In order to evaluate its performance, four items were placed in front of the AGV at different distances ranging between 80-300 centimeters away. The objects themselves were in different shapes and sizes to determine if taller objects would be identified by the LIDAR sensor with greater ease. 

% Lidar experiment 2
A second experiment was performed in order to test longer distances as well as different materials.
The AGV was placed in a long hallway with different surrounding walls, including transparent glass and bright white walls. Several full 360-degree rotations were made in order to see what the LIDAR picks up correctly.

\subsection{Driving surface test}
Different surfaces were used to test the driving accuracy and maneuverability of the AGV. Such surfaces include vinyl sheets, foam, artificial grass carpet, and wood. The AGV was made to drive straight for 2.25 meters on each of these surfaces.

